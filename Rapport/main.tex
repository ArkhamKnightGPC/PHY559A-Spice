\documentclass[a4paper,12pt,twoside]{article}
\usepackage{preamble}
\usepackage[titlepage,fancysections,pagenumber]{polytechnique}

\title{SPICE Electrical Simulation}
\subtitle{PHY559A}
\author{Gabriel Pereira de Carvalho}

\date{November 18, 2024}

\begin{document}

\maketitle

\tableofcontents

\newpage

\section{NMOS and PMOS transistors characterization}
{
	\subfile{sections/Q1.tex}
	\subfile{sections/Q2.tex}
}
\section{CMOS inverter static characterization}
{
	
	\begin{figure}[H]
		\centering
		\begin{circuitikz}
			% PMOS transistor
			\draw
			(0,3) node[pmos, anchor=D] (P1) {}
			(P1.S) -- ++(0,1) node[vcc] {$V_{DD}$}  % Vcc supply
			(P1.G) -- ++(-1,0) node[anchor=east] {$V_{in}$}  % Vin connection
			(P1.D) -- ++(0,-1) coordinate (mid);
			
			% NMOS transistor
			\draw
			(0,0) node[nmos, anchor=S] (N1) {}
			(N1.G) -- ++(-1,0) node[anchor=east] {$V_{in}$}  % Vin connection
			(N1.D) -- (mid) % Connect NMOS drain to PMOS drain
			(N1.S) -- ++(0,-1) node[ground] {};  % Ground connection
			
			% Output
			\draw (mid) -- ++(1,0) node[anchor=west] {$V_{out}$};
		\end{circuitikz}
	\end{figure}
	
	\subfile{sections/Q3.tex}
	\subfile{sections/Q4.tex}
	\subfile{sections/Q5.tex}
}
\section{Dynamical characterization of the CMOS inverter}
{
	\subfile{sections/Q6.tex}
	\subfile{sections/Q7.tex}
	\subfile{sections/Q8.tex}
}

\end{document}
